\chapter[Bartók's \textit{Romanian Folk Dances}, Sz. 56, BB 68]{Béla Bartók - \textit{Romanian Folk Dances}, Sz. 56, BB 68 (1915)}

Béla Bartók (1881-1945) was a Hungarian ethnomusicologist\footnote{This involves the study of music of world culture, both of the past and present, with an emphasis on the cultural, racial, and other influences, and affects on the music.}, pianist, and composer. Bartók was born in the Austro-Hungarian Empire, in a small Hungarian city in what is present-day Romania. With parents who were teachers and amateur musicians, Bartók went on to study music at the Hungarian Royal Conservatory of Music \autocite{Burkholder_Grout_Palisca_2014}. He returned to the Conservatory in 1907 to teach, but withdrew from public musical life in 1912\autocite{Gillies}. As a teacher, Bartók did not create a distinctive ``school'' for his students, as other teachers had, disinterested in teaching piano technique or other methods. In that time, Bartók only contributed and co-authored almost 50 piano pieces (Zongoriaskola, ``Piano Method'') in 1913, for piano technique at the school\autocite{Gillies}. These pieces began his ethnomusicological studies which would become his primary method of professional musical progress and income over the next several years. Later, Bartók began searching for innately Hungarian music, leading him to collect and study peasant music from many places. Bartók would go on to publish many songs and dances, with roots in Hungary, Romania, Slovakia, Croatia, Serbia, and Bulgaria, from his travels through central Europe, Turkey, and North Africa \autocite{Burkholder_Grout_Palisca_2014}. Within the new field of ethnomusicology, Bartók edited various collections and wrote books and articles which would establish him as a leading scholar in the field. To him, Hungarian peasant music represented the music of the country better than other types of music to come out of Hungary. This view was radical for a short time, as Hungary was ruled by an urban, German-speaking elite class, but eventually prevailed \autocite{Burkholder_Grout_Palisca_2014}. Based on peasant songs and dances, he created pieces, imitating the melodies, rhythms, and other characteristics found, blending them with more classical elements from his time at studying classical music. 

Around 1908, Bartók achieved a distinctive personal style, as his compositions introduced a new aspect to the piano. It was treated as a percussive instrument rather than a melodic one \autocite{Burkholder_Grout_Palisca_2014}. When World War I (WWI) began in 1914, Bartók's travels to collect folk music became near impossible \autocite{Gillies}. Instead, Bartók turned to arranging and creating folk-based music. In 1915, the Román nepi táncok (``Romanian Folk Dances'', BB 68) was composed, and in 1918, two Hungarian piano sets--Tizenöt magyar parasztdal (``15 Hungarian Peasant Songs'', BB 79), and ``Three Hungarian Folk Tunes'', BB80b--were finished \autocite{Gillies}. In the decade after WWI, Bartók pushed towards the previously-defined limits of musical dissonance and tonal ambiguity. In his synthesis of peasant folk music with classical music, there became an emphasis on the characteristics the two styles shared, and the distinctions each had. In both peasant folk songs and classical music, pieces will typically have a tonal pitch center, and will use diatonic and other scales, with melodies built from repeated motives \autocite{Burkholder_Grout_Palisca_2014}. From the classical side, Bartók pulled on contrapuntal and formal structures and procedures, with the fugue and sonata as examples. Then, from the folk songs, he pulled the irregular meters and complex rhythms, mixed modes and modal scales, and the defining melodic structure and specific ornamentation to folk songs \autocite{Burkholder_Grout_Palisca_2014}. Music by Bartók thus became simultaneously complex in counterpoint--more than Bach's-- and rhythmically complex with more ornamentation than the folk songs he referenced. The dissonance and harmonic strategies used were also based on the mixed concepts from the two musical styles. For example, Bartók frequently uses second and fourth chords\footnote{In a triadic chord, these chords would include the tonic, second, and fifth, and then the tonic, fourth, and fifth notes of the chord, respectively.}. Aesthetically, from a classical music perspective, Bartók placed himself somewhere in between Beethoven (in terms of artistic and harmony style), and Bach (in counterpoint). Later, in 1926, the styles of Schoenberg--expressionism\footnote{A trend which is common in the twentieth-century, in which composers attempt to convey emotional value and meaning through music, above all else. Musically, expressionism will often include a higher level of dissonance than non-expressionist music, extreme dynamic contrast, angular melodies with large leaps, more to bring the intended emotion across to the listener.}-and Stravinsky--neoclassicism\footnote{A trend also commonly found in the twentieth-century, in which composers sought to return to the ``golden age'' or so of Classical-era music. This return would be to the aesthetic characteristic associated with the Classical era, including balance, clarity, and emotional restraint.}--were also emulated \autocite{Gillies}.

\section{Romanian Folk Dances, Sz. 56, BB 68}

This mixture between elements of peasant folk dances from Romania, Hungary, Bulgaria, and other countries, and traditional classical music is clear in Bartók's ``Romanian Folk Dances''. The original name for ``Romanian Folk Dances'' included the phrase ``From Hungary'' at the end, as the pieces are based on folk songs Bartók heard during his time traveling to Transylvania \autocite{Burkholder_Grout_Palisca_2014}. The title was later changed when Transylvania became part of Romania in 1920, dropping the ending ``From Hungary''. Each piece here will be referred to its most well-known title (in Romanian), with the English translation in parentheses.

\subsection{Jocul cu bâtă (``Stick Dance'')}

This piece, titled ``Jocul cu bâtă'' or ``Stick Dance'' in English, is a stick dance meant to be energetic and upbeat. As the longest piece of the six in this set, this dance was performed by men solo, as a solo dance which consists of kicking the ceiling of the room in which the dance takes place \autocite{Weissmann_1969}. These details of the dance are clearly seen in the melody. In of itself, the melody, as in  in Figure \ref{fig:bartok-stick-dance-first-line}\autocite{Lung_2016}, is not as danceable as the title suggests. Instead, the melody is more elegant in sound, with the rhythm not lending itself easily to a dance-like sound. In the first line of the dance in Figure \ref{fig:bartok-stick-dance-first-line}\autocite{Lung_2016} the rhythm of the right-hand stutters in a syncopated melody. These articulations to the melody convey to the listener that this piece is a specific type of dance not commonly found, and to the performer the specific stylistic traits found within a stick dance. In the score, as in \ref{fig:bartok-stick-dance-first-line}\autocite{Lung_2016}, Bartók marks the articulations clearly. Within the first line itself, there are dynamic markings (the \textit{forte} and crescendo denoting that the volume of the beginning starts loud, and gets louder), and phrasing markings, detailing which notes should be connected to each other. Later in the piece, there are also other markings for the stick dance's intended style: accent marks, the usage of \textit{sforzando}, \textit{tenuto} and more. Bartók carefully guided the performer to the specific idea which he wanted the music to contain, doing so by also including the specific metronome tempo marking for the quarter note (80 beats per minute would be the equivalent to one quarter note, as seen in Figure \ref{fig:bartok-stick-dance-first-line}\autocite{Lung_2016}).

\begin{figure}[h]
  \centering
  \includegraphics[width=\textwidth]{figures/bartok-stick-dance-first-line.jpg}
  \caption{Béla Bartók, Romanian Folk Dances, \textit{Jocul cu bâtă}, mm. 1-4}
  \label{fig:bartok-stick-dance-first-line}
\end{figure}


Structurally, the piece is in binary form. In the A section, which lasts from bars 1-15, the first four bars (Figure \ref{fig:bartok-stick-dance-first-line}\autocite{Lung_2016}) feature a clear suggestion Bartók wanted. In this first phrase, the first full bar features two sixteenth notes played staccato, followed by a quarter note slurred into an eighth note. This phrase has the dynamic \textit{forte}, signaling to the dancer of the piece that the music is starting, and uses the staccato of the first two notes to signify that this piece will be energetic. As opposed to the rhythms found in the B section, this section is gentler, and more dance-like, but only in comparison to the B section. 

The B section of the piece begins in measure 15. As in Figure \ref{fig:bartok-stick-dance-b-section}\autocite{Lung_2016}, the rhythm changes to include dotted rhythms, marked here in red brackets. The first full measure of this bracketed section features one set of sixteenth note triplets. These notes are meant, as Bartók notates, to sound as ornamentation and support for the dancers, instead of a rhythmic tool solely for the instrumentalist. The set of triplets is slurred with the dotted quarter note before it, and the dotted eighth and sixteenth notes after it. The dotted rhythms of the eighth and sixteen notes which appear later in the B section may imitate the sounds of the dancers hitting the ceiling with the stick.

\begin{figure}
  \centering
  \includegraphics[width=\textwidth]{figures/bartok-stick-dance-b-section.jpg}
  \caption{Béla Bartók, Romanian Folk Dances, \textit{Jocul cu bâtă}, mm. 14-19}
  \label{fig:bartok-stick-dance-b-section}
\end{figure}

\subsubsection{Move Your Feet}

As a dance that is meant to emulate the actions of people stomping their feet while dancing and kicking the ceiling, I perform this dance with similar energy. At the beginning of the dance, as in Figure \ref{fig:bartok-stick-dance-first-line}\autocite{Lung_2016}, the left-hand's entrance is akin to the stomping of feet. It opens with a heavy rhythm of two quarter notes in the left-hand, with a \textit{tenuto}, defined as a directive in which the performer is meant to perform the indicated note or chord in a sustained manner for a longer duration than is typically assigned to the note or chords. Meanwhile, the right-hand plays an elegant melody which combined with the left-hand harmony does not lend itself easily to a dance. However, to me, as the performer, it is clear that instead of a typical danceable piece, this is a niche dance, with traits specific to the stick dance. In the first line of the dance, the dynamic level starts out \textit{forte}, and while the left-hand is played loudly, the right-hand is played even louder. This is helped by the staccato and carefully-placed legato lines in the right-hand, assisting my performance to begin the piece as loud and dramatic. In the last beat of the first full measure of Figure \ref{fig:bartok-stick-dance-first-line}\autocite{Lung_2016}, a crescendo begins, so I build even more energy into my performance. The dancers of a stick dance would not allow the climax and the height of their energy to peak at such an early point in the dance, and so neither do I in my interpretation. I still contain energy to emulate the idea of feet rapidly stomping the ground while dancing, and sticks hitting the ceiling. 

When the B section begins, this is the point in my performance where I imagine the sticks of the dance begin to hit the ceiling. As in Figure \ref{fig:bartok-stick-dance-b-section}\autocite{Lung_2016}, notes in higher octaves begin to be used more frequently. With these higher notes, I place emphasis onto passages such as measure 20, in which a sixteenth note which is slurred into an eighth note (along with the eighth note's staccato accent). Furthermore, the addition of the dotted eighth and sixteenth notes, which appear later in the B section, add to the atmosphere of sticks hitting the ceiling during the dance. I play these rhythms shorter than I play the dotted rhythms which appear elsewhere in the piece, to further emphasize the sticks hitting the ceiling, while the dancers continue at a fast pace. 

\subsection{Brâul (``Sash Dance'')}

The second of the dances in the set, Brâul (``Sash Dance''), is the shortest dance, at only two lines long. A Brâul is a typical dance out of Romania, in which traditionally a sash or other type of waistband is used as an accessory in the dance. It is a fast and energetic dance, with an undertone of sadness to the tune. This undertone is reflected in the freedom Bartók has given the performer, as in Figure \ref{fig:bartok-waistband-dance-interpretation}\autocite{Lung_2016}. The only non-rhythmic aids Bartók gives the performer is a metronome tempo marking for the quarter note, and a note to repeat the whole piece again at the end, introducing a slow \textit{ritardando} at the end. Though there is no key signature, we see that through the first bar of the piece, the left-hand signals that the tonic key of the piece will be D Minor. At the beginning of the piece, it is already energetic, with the \textit{allegro} tempo marking, and the quarter note equal to 144 beats per minute (Figure \ref{fig:bartok-waistband-dance-interpretation}\autocite{Lung_2016}). In its rhythm, each note is mostly to be played \textit{staccato}, with only a few notes played slurred, or held out with a note longer than the length of a quarter note. The melody matches this bouncy feel which the rhythm gives the piece. After each phrase, the motion of the piece as a whole stops for a moment. As in Figure \ref{fig:bartok-waistband-dance-interpretation}\autocite{Lung_2016}, there is clear, continuous motion as the melody continues on. But, as the left-hand's rhythm becomes smoother to signal a shift to a longer note, so does the melody, containing a thirty-second note before coming to a temporary close on a half-note. 

\begin{figure}[h]
  \centering
  \includegraphics[width=\textwidth]{figures/bartok-waistband-dance-tempo-marking-and-dynamic.jpg}
  \caption{Béla Bartók, Romanian Folk Dances, \textit{Brâul}, mm. 1-4}
  \label{fig:bartok-waistband-dance-interpretation}
\end{figure}

\subsubsection{The Importance of a Sash}

This is the shortest dance in the six-dance set, so I do not lose the intended fast-pace energy once. This dance is flexible in rhythm, up to the performer's interpretation, as the only thing that I must keep in mind that the dance is meant to be performed \textit{allegro}, with the quarter note equal to 144 beats per minute. Knowing this, I begin the piece straightforward in rhythm, as if I were to stick to what Bartók had written exactly. The right-hand melody starts at \textit{piano}, and is staccato, while the left-hand quarter notes are both played staccato and use the pedal. But, with the dance's beginning two sounds become clear: the stomping of the feet of the dancers in the left-hand, and the sash, which is used as an accessory, in the right-hand. I also allow the left-hand's harmony to become background noise, as actually foot movements would be, had the dance taken place in-person. I emphasize the right-hand's melody, which is made easier by the staccato notes and occasional quick sixteenth note rests. For the second time playing through this piece, which is marked by the \textit{ossio: 2 con 8va} (a directive to raise the pitches played by the right-hand by an octave upon the second play through) additional energy is brought to the dance, through the raised right-hand notes. 

% dance 2
% talk about the middle section being the "sad" part of the dance, as there's that one part in the middle which is different and only slightly less upbeat, the dance itself is really start, and stop, make something up?

\subsection{Pe loc (``In One Spot'')}

The music of the second dance, ``Brâul'', flows directly into the third dance, ``Pe loc'' (``In One Spot''). It is much slower than ``Brâul'', giving a greater contrast. The beginning of the piece, as in Figure \ref{fig:bartok-one-spot-beginning}\autocite{Lung_2016}, features an overall slower melody and accompaniment. The range that the right-hand's melody plays is confined to small intervals, narrow enough to fit within the one high octave in which it is played. It is also meant to be played slowly, marked with the \textit{Andante} tempo marking, and the metronome marker of 108 beats per minute for the quarter note. The accompaniment that is found in the left-hand is that of a droning sound, as the only movement within the piece is in the melody. This gives the dance the idea that it is to be performed in one spot, as both the right-hand's melody and left-hand accompaniment stay within the range of an octave.

\begin{figure}[h]
  \centering
  \includegraphics[width=\textwidth]{figures/bartok-one-spot-beginning.jpg}
  \caption{Béla Bartók, Romanian Folk Dances, \textit{Pe loc}, mm. 1-6}
  \label{fig:bartok-one-spot-beginning}
\end{figure}

The dance is in binary form, with sections A (measures 1-18) and B (measures 19-40). The A section starts the dance in the key of B Minor, but also features the use of the raised fourth scale degree, with the E\musSharp{} instead of E, as seen in Figure \ref{fig:bartok-one-spot-beginning}\autocite{Lung_2016}, marked in red. When E\musSharp{} is played before or after the note D, the augmented second interval is created. This augmented second interval--in which the typical one whole tone distance between one note and the next is increased by a semitone, and the resulting interval containing the distance of one and a half tones--is what creates the gypsy-like, rooted sound quality, containing the interval typical of many Middle Eastern instrumental pieces, the augmented second. The B section shifts the tonal center of the dance, modulating to D Major, the relative major key to B Minor, the tonic key found in the A section. This is found in Figure \ref{fig:bartok-one-spot-b-section}\autocite{Lung_2016}, where the left-hand accompaniment shifts to center around the chord of D Major. Similar motivic material, for both melody and harmony, is used as it was found in the A section.

\begin{figure}[h]
  \centering
  \includegraphics[width=\textwidth]{figures/bartok-one-spot-b-section.jpg}
  \caption{Béla Bartók, Romanian Folk Dances, \textit{Pe loc}, mm. 19-23}
  \label{fig:bartok-one-spot-b-section}
\end{figure}

\subsubsection{Snake Charming}

The third dance starts slow, with the \textit{Andante} tempo marking, and one quarter note equal to 108 beats per minute. To me, this indicates that this dance is meant to be mysterious in sound. Like with other dances in the set, this dance's melody is also confined to be within one octave, adding to the secretive sound. At the beginning of the dance, the left-hand is sounding quarter notes, a dyad, then a single quarter note. This becomes a droning sound that the right-hand's melody is able to to sing out above, so while the dynamic marking is \textit{piano}, I play the right-hand at a \textit{mezzo forte} level. 

It is this combination of interpretative tools, along with those previously mentioned, that causes me to correlate the sounds of a dance which is performed in a singular spot to the stereotypical ``snake charmer'' song to which United States-based audiences are familiar. This sound, known as the ``Arabian riff'' is a melody which has first been published in the nineteenth-century by Sol Bloom, a show promoter and later U.S. Congressman during the 1893 Chicago World's Colombian Exposition\autocite{Shira}. One attraction of the 1893 Exposition was called ``A Street in Cairo'' which featured snake charmers, a dancer named ``Little Egypt,'' and other Arabian and African features. The earliest known recording was created in 1895, under the name ``Streets of Cairo or The Poor Little Country Maid.'' For United States audiences especially, this melody is used to display connections to the Arabian Peninsula, which includes Iran (Persia), India, Egypt, and images of belly dancers, camels, snake charmers, or other similar ideas \autocite{Shira}.

As seen in measures 4 and 5 of James Thorton's ``Streets of Cairo or The Poor Country Maid'', in Figure \ref{fig:thorton-streets-of-cairo}\autocite{Thorton_1895} as well, the use of the major second interval is what gives this riff part of its signature sound. ``Pe loc'' does the same, as measure 4 of Figure \ref{fig:bartok-one-spot-beginning}\autocite{Lung_2016} uses the note E\musSharp{}, which is the raised scale degree two, or an augmented second interval. ``Streets of Cairo'' has the same mysterious sound quality as ``Pe loc,'' which I emulate, allowing the right-hand melody to sing above the droning of the left-hand when the pitches of the melody rise. Thus, the influences of Bartók's trips to Northern African shine through, especially in the B section of the dance, like in Figure \ref{fig:bartok-one-spot-b-section}\autocite{Lung_2016} in which sixteenth notes and eighth notes are played in some combination of legato and staccato.

\begin{figure}
  \centering
  \includegraphics[width=\textwidth]{streets-of-cairo.jpg}
  \caption{James Thorton, ``Streets of Cairo or The Poor Country Maid,'' mm. 4-5}
  \label{fig:thorton-streets-of-cairo}
\end{figure}


% dance 3
% I think of a snake charmer, it's not very subtle, but it's alluring -> this is both alluding to the gypsy-like quality of the song (and as Romania was "known" to have more gypsies back then (back this point up some how), as well as the mythical quality of snake-charmers
% snake-charmer songs don't move from the one-spot, so because of Middle Eastern influences ->  The
% melody’s accents, together with the ornamentation, syncopated rhythms, and the change
% of the key center in the “B” section, are elements that could be seen in the music as a
% sound representation of the variety of visual gestures done by the dancer to make the
% dance attractive

\subsection{Buciumeana (``Dance from Bucsum'')}

The fourth dance in this set, ``Buciumeana'', comes from modern day Bucium, in Alba county in Romania. It is meant to be played slow, as evident by Figure \ref{fig:bartok-dance-four-first-line}\autocite{Lung_2016}, in which the given tempo marking is \textit{Moderato}, with quarter notes to be played at 100 beats per minute. Because it is slow, it has a childlike simplicity to it. Both rhythm and melody are simple in sound, with the left-hand accompaniment treating solely quarter notes and quarter rests. An important feature of this dance to notice, as in Figure \ref{fig:bartok-dance-four-first-line}\autocite{Lung_2016}, is that in the right-hand's melody, the shape of the melody is downwards. Every phrase in the melody goes in a downwards direction, giving the dance an almost wistful and youthful sound. This dance is constructed in binary form, with section A from measures 1-10, and section B from measures 11-18. Both sections are in the piece's tonic key of A Major, but treat slightly different thematic material. 

\begin{figure}[h]
  \centering
  \includegraphics[width=\textwidth]{figures/bartok-dance-four-first-line.jpg}
  \caption{Béla Bartók, Romanian Folk Dances, \textit{Buciumeana}, mm. 1-5}
  \label{fig:bartok-dance-four-first-line}
\end{figure}

At the beginning of the dance, in addition to the tempo markings previously mentioned, we notice that this piece is to be played in $\frac{3}{4}$ time. This is the first dance of the set to be in $\frac{3}{4}$. However, the feeling of $\frac{3}{4}$ time is not easily felt in the first three bars of the dance. As in \ref{fig:bartok-dance-four-first-line}\autocite{Lung_2016}, the accompaniment features tied quarter notes between bars. Within the first two bars, the quarter note of the last beat of bar 1 is tied with the quarter note of the first beat of bar 2. This causes the understanding of the $\frac{3}{4}$ time signature to be unclear, until the melody enters in bar 3. Through the use of these tied notes, it will be difficult for the listener to distinguish the proper time signature of $\frac{3}{4}$ from the one they hear in the opening. The listener would hear the first two measures of the dance as being in a $\frac{2}{4}$ time signature instead, resulting in brief metric displacement for the listener. The B section of this dance introduces the beginning of sixteenth note phrases slurred together, and the feeling of more motion happening than in the A section. These long legato lines through the phrases which feature sixteenth notes in the melody and simpler quarter notes in the harmony, as seen in Figure \ref{fig:bartok-dance-four-b-section-two-lines}\autocite{Lung_2016}, convey a thoughtful sound. The long phrases with legato lines are paired with a descent in the melody. 

\begin{figure}[h]
  \centering
  \includegraphics[width=\textwidth]{figures/bartok-dance-four-b-section-two-lines.jpg}
  \caption{Bela Bartók, Six Romanian Folk Dances, \textit{Buciumeana}, mm. 11-15}
  \label{fig:bartok-dance-four-b-section-two-lines}
\end{figure}

The similarities between sections A and B begin after the first two bars of each section. The themes in the sections become visible, as seen in figures \ref{fig:bartok-dance-four-first-line}\autocite{Lung_2016} and \ref{fig:bartok-dance-four-b-section-two-lines}\autocite{Lung_2016}. It is only the first two measures of each section which differentiates them. The first two bars of section A begin with a half note and quarter note, followed by an eighth note phrase, while the first two bars of section B begin with a phrase of sixteenth and eighth notes. After these two measures, Bartók uses the same melodic material between the two sections. 

\subsubsection{Modern Day Bucium}

The fourth dance in the six-dance set is played at a medium tempo: \textit{Moderato} in which the quarter note is to be played equivalent to 100 beats per minute. There is a childlike wonder to this dance to me, so I perform it as if it is meant to be a child dancing this piece. The first three measures of the dance, in Figure \ref{fig:bartok-dance-four-first-line}\autocite{Lung_2016}, are played with the pedal held down, and is note released until the chords played in the left-hand change. The piece begins at \textit{piano}, and the left-hand provides a simple backdrop which the right-hand is able to sing above, at a dynamic level of \textit{mezzo forte}. The phrase \textit{molto expressivo} (notated as \textit{molto espr}) also allows me to bring forth a childlike wonder to the dance. I play with the dynamic level A section; when the melody rises in pitch, I increase the dynamic level to \textit{mezzo forte}, and when the melody descends in pitch I lower the dynamic level back to \textit{piano}. Combined with motion in the right-hand, I perform the section as wistful, and a child imagining something fantastical.

The childlike wonder within the piece is more evident in the B section, when sixteenth notes are introduced in the melody line. As in Figure \ref{fig:bartok-dance-four-b-section-two-lines}\autocite{Lung_2016}, the sixteenth notes are grouped into twos, which in contrast to the quarter notes in the left-hand create a bouncy type of motion. 

% dance 4 
%The two agents are described in the first two measures of every phrase (in the “A”
%section, the order character mm. 3-4, 7-8 and in the “B” section the transgressor mm. 11-
%12, 15-16, due to the unexpected rhythmic change). The conflict between the two results
%from the rhythmic differences (see above Example 14). The order imposing character has
%triplets in its construction and dotted rhythms which makes it contrasting in itself due to
%the instability between the triplets and dotted notes. The transgressor has running
%sixteenth notes, syncopation over the bar which makes it feel weaker, and dotted rhythms.
%Even if the last phrase of the piece starts with the thematic idea of the transgressor,
%Bartók reminds the listener about the order imposing character with the last two measures
%of the phrase. This places the piece in the narrative archetype of Romance meaning that
%the order character is fulfilling its objective over the one that wanted to interfere.
%Because of this rhythmic instability between the characters, the performer has the
%opportunity to express varied nuances of yearning or longing which are characteristic to
%Romanian music. The fact that the piece ends with the same two measures that end every
%phrase up to that point it can be interpreted as recurrent obsessive statement

\subsection{``Poarga Românească'' (Romanian Polka)}

The fifth dance, ``Poarga Românească'', comes from modern day Beius, on the border between Hungary and Romania. It is reminiscent of an old Romanian dance similar to the Polka. It is notable as being the only dance in this set with a consistent alternating meter, between $\frac{3}{4}$ and $\frac{2}{4}$, in a pattern of two measures in $\frac{3}{4}$, followed by one measure in $\frac{2}{4}$, as in Figure \ref{fig:bartok-dance-five-time-signature}\autocite{Lung_2016}. This hypermeter (defined as groups of measures which form patterns of accentuation, especially at faster tempos\autocite{Hughes_Gotham_Hamm_2021}) creates a sense of asymmetry, which results in an energetic, continuous-movement in the dance, and a three-measure phrase. The last bar of this three-bar phrase is much shorter than the first two. The piece overall contains an energetic peasant-like dance theme, played in total for about a half-minute in length. The dance starts in the key of D Major, in Figure \ref{fig:bartok-dance-five-first-four-bars}\autocite{Lung_2016}, with the raised fourth scale degree G\musSharp{} (Figure \ref{fig:bartok-dance-five-time-signature}\autocite{Lung_2016}).

\begin{figure}[h]
  \centering
  \includegraphics[width=\textwidth]{figures/bartok-dance-five-time-signature.jpg}
  \caption{Béla Bartók, Romanian Folk Dances, \textit{Poarga Românească}, mm. 7-10}
  \label{fig:bartok-dance-five-time-signature}
\end{figure}

\begin{figure}[h]
  \centering
  \includegraphics[width=\textwidth]{figures/bartok-dance-five-first-four-bars.jpg}
  \caption{Béla Bartók, Romanian Folk Dances, \textit{Poarga Românească}, mm. 1-4}
  \label{fig:bartok-dance-five-first-four-bars}
\end{figure}

Structurally, this dance is in binary form. Section A lasts from measures 1-16, and section B from measures 17-28. In the A section, the usage of \textit{staccato} and grace notes are two articulation phrases which contribute to the update, danceable nature of the piece. The \textit{staccato}, and thus the sudden bounciness of the note, adds a sense of quickness and dancing energy. The grace notes, as in Figure \ref{fig:bartok-dance-five-time-signature}\autocite{Lung_2016}, also contribute to the high-energy of the piece, with \textit{staccato} notes and slurred notes. In the B section, Bartók introduces syncopation in the left-hand's accompaniment, as in Figure \ref{fig:bartok-dance-five-b-section}\autocite{Lung_2016}. A \textit{szforzando} is added to the end of each phrase as a clear ending, such as in the last bar in Figure \ref{fig:bartok-dance-five-b-section}\autocite{Lung_2016}. Then, from bar 20 through to the end of the dance, Bartók writes a \textit{szforzando} every third measure, and in bars 25 and 28, includes a szforzando in the first and third beats of the measure (Figure \ref{fig:bartok-dance-five-b-section}\autocite{Lung_2016}), creating a contrast with the beginning of the piece. 

\begin{figure}
  \centering
  \includegraphics[width=\textwidth]{figures/bartok-dance-five-b-section.jpg}
  \caption{Béla Bartók, Romanian Folk Dances, \textit{Poarga Românească}, mm. 16-19}
  \label{fig:bartok-dance-five-b-section}
\end{figure}

\begin{figure}
  \centering
  \includegraphics[width=\textwidth]{figures/bartok-dance-five-ending.jpg}
  \caption{Béla Bartók, Romanian Folk Dances, \textit{Poarga Românească}, mm. 20-28}
  \label{fig:bartok-dance-five-ending}
\end{figure}

\subsubsection{The Polka}

Visually, the fifth of the six-dance set appear similar to a polka. To dance a polka, the dancer must be aware of the rhythm of polka. As a fast two-step, the first beat is accented, and the second beat is not. Typically, the basic step of a polka will include three steps in a quick, quick, slow rhythm, and this is the same visually as in Figure \ref{fig:bartok-dance-five-first-four-bars}\autocite{Lung_2016}, in which the first beat of measure 1 is accented by a quarter note with a tenuto mark, and beat 2 is not accented, with two eighth notes played staccato. Later, in the B section, through the rhythm of both hands, I create the perception of the strong beat landing on beat 1, and the weak beat on beat 2, of Figure \ref{fig:bartok-dance-five-b-section}\autocite{Lung_2016}. Though neither the right-hand melody nor left-hand harmony themselves contain a clear strong downbeat on beat 1, the combination does. 

\subsection{``Mărunțel`` (Fast Dance)}

\begin{figure}
  \centering
  \includegraphics[width=\textwidth]{figures/bartok-dance-six-b-section-syncopation.jpg}
  \caption{Béla Bartók, Romanian Folk Dances, \textit{Mărunțel}, mm. 20-25}
  \label{fig:bartok-dance-six-b-section-syncopation}
\end{figure}


The last dance of the set of six Romanian Folk Dances is ``Mărunțel``, which features two distinct yet similar themes. Each of these themes feels breathless in pacing, as a celebration of some kind without urgency. The beginning of the dance features a narrow melody, each of phrase of which is contained within an octave and uses small steps (Figure \ref{fig:bartok-dance-six-first-line}\autocite{Lung_2016}). As a whole, the piece is written in binary form, with section A lasting from measures 1-16, and section B from measures 17 to the end. The tonic of the piece begins in D Major, and uses a raised fourth scale degree\footnote{We should now be able to recognize the raised fourth scale degree as a feature typical of Romanian folk dances.} G\musSharp{}. At the beginning of the A section, we notice that the melodic line embodies the title of the piece (which in English translates to ``Fast Dance''), which fast melodic and harmonic phrases, as in Figure \ref{fig:bartok-dance-six-first-line}\autocite{Lung_2016}. The fast sixteenth note phrases of the section represent the fast steps dancers would need to perform to keep to the beat of the dance, but must also not be overplayed by the instrumentalist. There are also accents on also every beat to emphasize the $\frac{2}{4}$ time signature of this dance, and to provide balance against the rhythm of the harmony. When the B section begins, there is a change to the tempo, from \textit{Allegro} to the \textit{Più allegro}. There is also an introduction of triplets, which contributes to the higher-energy feeling of the B section. When the dance is played at tempo, these triplets sound as ornamentation, due to the speed at which they are played, as circled in blue in Figure \ref{fig:bartok-dance-six-b-section}\autocite{Lung_2016}. 

\begin{figure}
  \centering
  \includegraphics[width=\textwidth]{figures/bartok-dance-six-first-line.jpg}
  \caption{Béla Bartók, Romanian Folk Dances, \textit{Mărunțel}, mm. 1-6}
  \label{fig:bartok-dance-six-first-line}
\end{figure}

\begin{figure}
  \centering
  \includegraphics[width=\textwidth]{figures/bartok-dance-six-b-section.jpg}
  \caption{Béla Bartók, Romanian Folk Dances, \textit{Mărunțel}, mm. 14-19}
  \label{fig:bartok-dance-six-b-section}
\end{figure}

The B section is twice as long as the A section, with the melodic material which begins the B section returning in measure 33. However, the material does not return as an exact duplication of the first treatment of the material. There are variations to the left-hand harmony's chord progression, and Bartók eliminates the syncopated rhythm in favor of using the smoother-sounding quarter and eighth notes. By simplifying the accompaniment, the listener can focus on the melody.

\subsubsection{A Courtship}

There are two distinct themes to the last of Bartók's six dances, the ``Mărunțel.`` As a fast dance, this could be considered to be a courtship dance between two people. Each theme feels breathless in its tempo and articulation, with celebratory undertones. Within the A section, there is narrow melody, which encapsulates only one octave and no consecutive jumps larger than the interval of a third. It immediately captures the feeling the title states (``Fast Dance''), as the tempo marking for the section has one quarter note equivalent to 146 beats per minute. The person who the A section represents begins the piece fast and ready. Two accented notes, on D in the left-hand and A in the right-hand as in Figure \ref{fig:bartok-dance-six-first-line}\autocite{Lung_2016}, start the dance in \textit{forte}. The notes played staccato bring energy, while the notes that are played legato, and the pedal, do not subtract from the total energy in the piece. I mirror this, allowing the left-hand's dyads to bring the same energy as the narrow melody in the right-hand does. The power of the left-hand chords allow the right-hand's melody to stand out even more against the backdrop of motion, as the difference between the sixteenth notes and eighth notes in the right-hand and the eighth notes and quarter notes in the left-hand is clear. 

With the start of the B section, the energy and speed that was heard in the A section does not disappear; the energy of the B section includes more motion that involves syncopation and triplets. As in Figure \ref{fig:bartok-dance-six-b-section}\autocite{Lung_2016}, in the right-hand the triplets are circled in blue, which is the start of various syncopated rhythms found in the section. I try not to accent the syncopated notes themselves, but accent syncopation that falls on the strong beats of 1 and 2 in the section. Syncopation in the right-hand adds ornamentation to the piece, due to the speed at which it is played. The person represented by the B section ``dances'' twice as longe than the person who is represented by the A section (that is, the B section is twice as long as the A section), but the energy which I put into the B section does not fade. There is also a tempo change from \textit{Allegro} to \textit{Piú allegro}, in which person B is dancing faster than person A.