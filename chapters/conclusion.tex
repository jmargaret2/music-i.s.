\chapter*{Conclusion}\label{conclusion}
\addcontentsline{toc}{chapter}{Conclusion}

\section*{A Performer's Interpretation}
\addcontentsline{toc}{section}{A Performer's Interpretation}

\subsection*{J.S. Bach and \textit{The Well-Tempered Clavier, Book I}}

\subsection*{Mozart and \textit{Piano Sonata in C Major No. 10, K 330}}

\subsection*{Beethoven and \textit{Eleven Bagatelles Op. 119 No. 1}}

\subsection*{Chopin and \textit{Prelude, Op. 28, No. 15}}\label{subsection:chopin-intepretation}

Chopin: Charon (the one who leads Greeks to the underworld) coming \& person is scared of death. Therefore first section is the looming wait, second section is he’s here. Third section could be interpreted as acceptance or willingly going with Charon, or the section B2 is about the fight and fear of death/Charon, and the return of the first melody is his acceptance & willingness to go along. (Insert Greek explanation of Charon/death)

Could also tie in idea of rain and the literal rain interpretation with the intense fear of death \& eventual acceptance with the calm after a heavy rain \& stereotypical rainbow afterwards --> this version would be because he's literally writing, health isn't good, \& wife/family are not on the island he's on they're somewhere else

\subsection*{Tchaikovsky and \textit{The Seasons, Op. 37a}}

Within these first five bars, we also take care to notice that the left hand's rhythm is reminiscent of a boat rocking

Talk about how it's a character piece, so therefore is literally pulling in the motion of being on a gondola \& the Venetian gondolier singing

Gloom atmostphere, also about death?

Mix of happiness and sadness?

Ending of the piece from bars 92-end we see the left hand be super two-note-y type where it's only one note then another, and not rocking motion like it was before. This could literally simulate the end of the boat ride, as the passengers disembark as the gondolier finishes singing.

Insert some weird interpretation about death /& acceptance.

\subsection*{Bartok and \textit{Romanian Folk Dances, Sz. 56, BB 68}}