\chapter*{Conclusion}\label{conclusion}
\addcontentsline{toc}{chapter}{Conclusion}

A year ago, I began to prepare for my senior recital. Throughout the second half of my junior year spring semester and into the summer, I researched composers and various piano pieces, and the historical and harmonic significance each piece contained, to become comfortable with the repertoire. I spent hours listening to arrangements of the pieces on repeat, variations of interpretations, and the rhythmic liberties allowed in the pieces which were composed during the Romantic era and later. The research that I have conducted over the past year has helped me understand the history of various recognizable pieces of each period of music, which in turn allowed me to uncover a deeper appreciation for the individual pieces, as well as how each piece related to the others. Weekly lessons with Dr. Yuka Nakayama-Lewicki encouraged me to find deeper meanings in even the most absolute of pieces, and to discover the kind of personal touch I would be able to bring to a performance. 

Through my one-on-one work with Dr. Nakayama-Lewicki, I was in an environment where I began to experiment freely with expression in pieces from the Romantic and Contemporary eras of music, and technique in pieces of the Baroque and Classical periods. I explored the extra-musical nature of pieces by Chopin and Tchaikovsky, which elegantly painted a picture of an event happening in front of the listener; I uncovered the meaning behind a ``well-tempered clavier,'' and the ways in which it is possible to give emotion to Baroque or Classical pieces. As a performer, the solo recital is a tool that will help me build my confidence in my stage presence and musicality. After numerous ``practice'' recitals in the form of departmental recitals, I now feel more comfortable performing in a group setting on a stage. This senior recital repertoire has given me a new perspective and appreciation for piano performance strategies, methods through which to interpret Baroque era pieces, and the art of solo recitals.