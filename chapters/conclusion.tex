\chapter*{Conclusion}\label{conclusion}
\addcontentsline{toc}{chapter}{Conclusion}

\section*{J.S. Bach and \textit{The Well-Tempered Clavier Book I}}

\section*{Mozart and \textit{Piano Sonata in C Major No. 10}, K 330}

\section*{Beethoven and \textit{Eleven Bagatelles} Op. 119 No. 1}

\section*{Chopin and \textit{Prelude}, Op. 28, No. 15}\label{subsection:chopin-intepretation}

%Chopin: Charon (the one who leads Greeks to the underworld) coming \& person is scared of death. Therefore first section is the looming wait, second section is he’s here. Third section could be interpreted as acceptance or willingly going with Charon, or the section B2 is about the fight and fear of death/Charon, and the return of the first melody is his acceptance \& willingness to go along. (Insert Greek explanation of Charon/death)
%
%Could also tie in idea of rain and the literal rain interpretation with the intense fear of death \& eventual acceptance with the calm after a heavy rain \& stereotypical rainbow afterwards --> this version would be because he's literally writing, health isn't good, \& wife/family are not on the island he's on they're somewhere else

\section*{Bartok and \textit{Romanian Folk Dances}, Sz. 56, BB 68}

% https://tonycalifano.blogspot.com/2008/07/bartoks-romanian-folk-dances.html

% dance 1
%left hand comes in, simulating the stomping of feet while dancing as it opens with a heavy rhythm -> talk about this foot stomping rhythm -> resulting in a dance off of two random specific individual dancers
% Because in
%the entire B section there are no rhythmic motivic elements that could remind the listener
%about the A section, the story of the piece unfolds the idea that the second character is the
%stronger one. This understanding of the work and the elements that are building it should
%give the performer a perspective on how to shape the “personalities” of the two
%characters. In other words, the AB sections can represent two dancers that compete. The
%first one is obviously overtaken by the second one which presents a more energetic,
%flamboyant and dynamic dance, characterized by dotted rhythms, more varied
%articulation, larger range of dynamics.

% dance 2
% talk about the middle section being the "sad" part of the dance, as there's that one part in the middle which is different and only slightly less upbeat, the dance itself is really start, and stop, make something up?

% dance 3
% I think of a snake charmer, it's not very subtle, but it's alluring -> this is both alluding to the gypsy-like quality of the song (and as Romania was "known" to have more gypsies back then (back this point up some how), as well as the mythical quality of snake-charmers
% snake-charmer songs don't move from the one-spot, so because of Middle Eastern influences ->  The
% melody’s accents, together with the ornamentation, syncopated rhythms, and the change
% of the key center in the “B” section, are elements that could be seen in the music as a
% sound representation of the variety of visual gestures done by the dancer to make the
% dance attractive

% dance 4 
%The two agents are described in the first two measures of every phrase (in the “A”
%section, the order character mm. 3-4, 7-8 and in the “B” section the transgressor mm. 11-
%12, 15-16, due to the unexpected rhythmic change). The conflict between the two results
%from the rhythmic differences (see above Example 14). The order imposing character has
%triplets in its construction and dotted rhythms which makes it contrasting in itself due to
%the instability between the triplets and dotted notes. The transgressor has running
%sixteenth notes, syncopation over the bar which makes it feel weaker, and dotted rhythms.
%Even if the last phrase of the piece starts with the thematic idea of the transgressor,
%Bartok reminds the listener about the order imposing character with the last two measures
%of the phrase. This places the piece in the narrative archetype of Romance meaning that
%the order character is fulfilling its objective over the one that wanted to interfere.
%Because of this rhythmic instability between the characters, the performer has the
%opportunity to express varied nuances of yearning or longing which are characteristic to
%Romanian music. The fact that the piece ends with the same two measures that end every
%phrase up to that point it can be interpreted as recurrent obsessive statement

% dance 6 -> courtship dance between 2 people