\chapter[Tchaikovsky and \textit{The Seasons}, Op. 37a]{Pyotr Ilyich Tchaikovsky - June: Barcarolle from \textit{The Seasons}, Op. 37a (1875)}

Pyotr Ilich Tchaikovsky (1840-1893) was a leading Russian composer during the nineteenth century. Born in Votkinsk, Russia, Tchaikovsky moved with his family several times, including to Moscow and St. Petersburg, as his father searched for a job\autocite{Burkholder_Grout_Palisca_2014}. This did not dissuade Tchaikovsky from attempting to pursue a serious musical study, as he studied at the St. Petersburg Conservatory between 1862-1865, and then left to Moscow, where he received a teaching position at Moscow's Conservatory. During his time at the Moscow Conservatory, he composed his work \textit{The Seasons}, twelve character pieces for piano. A character piece is a type of program music. Program music is defined as a type of music which tells a story, able to illustrate various literary ideas or evoke pictorial scenes\autocite{Kennedy_Kennedy_Rutherford-Johnson_2013_programme_music}, and this definition can be expanded to include all music which attempts to represent some type of extra-musical material or concepts without spoken words\autocite{Scruton_2001}. The term ``program music'' was introduced by Franz Liszt, also the creator of the term \say{symphonic poem}, to describe the most characteristic aspect of program music. He described program music as music which the composer can both guard the listener against the ``wrong'' interpretation of the music, and also direct the listener's attention to the ``proper'' ideas of the piece as a whole or a particular piece of it. Contrasted with absolute music, music without these extra-musical concepts, program music is best characterized by its attempt to depict objects and events, not merely echoing or imitating these. In some cases, these pieces may attempt to literally evoke the scene of some type, with Tchaikovsky does to excellent effect with his June Barcarolle from \textit{The Seasons}. A character piece is designed to convey a specific allusion, atmosphere, mood, or scene, without text, stage action, or other program assistance\autocite{Temperley_2011}. The Romantic era (during the nineteenth century) encouraged literary influences in music, and nationalism of the time led composers, including Tchaikovsky, to evoke the folk music of various nations and ethnic groups. Tchaikovsky did this in his music by combining his Russian heritage with influences from Italian opera, French ballet, and Germany symphony and song, music similar to program music \autocite{Burkholder_Grout_Palisca_2014}.

Each piece in \textit{The Seasons} is of a different month of the year in Russia. The month of June is written in the style of a barcarolle. This is a song written in $\frac{6}{8}$ or $\frac{12}{8}$ time, typically sung by Venetian gondoliers, and has an accompaniment which suggests the rocking of a Venice gondolier\autocite{Latham_2011_barcarolle}. Within the June barcarolle, there is a slight deviation from the typical barcarolle piece. The piece is written in G Minor, and has an ABA form. Within the A section in Figure \ref{fig:t-first-three-lines}, we notice that this piece is actually in $\frac{4}{4}$ instead of the usual $\frac{6}{8}$ or $\frac{12}{8}$ time. This opening is marked as \textit{Andante cantabile}, or ``slowly sing'', to ensure the performer understands the sensitivity that must be brought to perform this piece. The time signature of $\frac{4}{4}$ and tempo marking \textit{Andante cantabile} gives a sense of pace, as well as the necessary combination of correct rhythm--as the Venetian gondolier ``sings''--and the treatment of the right hand's melody line. In the first two lines of the barcarolle, a dactylic (defined as a metrical pattern referring to a rhythm in which one strong syllable or beat is followed by two unstressed, or not strong, or short syllables or beats)\autocite{dactylic} pattern emerges, as clearly seen by Figure \ref{fig:t-first-three-lines}\autocite{Henle_2002}. Here, there is a clear long phrase which starts the dactylic pattern in bar two. The second beat of four through to the first three beats of bar six finish the first set of the dactylic pattern. This pattern repeats, in the last beat of bar six through to the first three beats of bar twelve. This twelve-bar phrase could form some type of period, ending of the first phrase group in the middle of bar six differs slightly from the ending of the second phrase group in the middle of bar twelve. Thus, there is an atypical form of the traditional period in the first twelve bars of the piece, found bracketed in red in Figure \ref{fig:t-first-three-lines}\autocite{Henle_2002}. 

Additional stylistic choices to create necessary sensitive and delicate sound, akin to physically being on a Venetian gondola and hearing the water and boat move, involve what is notated in Figure \ref{fig:t-a-section-style-choices}\autocite{Henle_2002}. First, there are several instances of a change in dynamics, with several crescendos and decrescendos marked, along with a \textit{piano} and \textit{forte} to obtain the right level of volume and balance from a performer. Specific to these measures is also the dynamic marker \textit{poco piú} (a marking meaning ``little more''), signifying a crescendo before a crescendo. The phrase peaks in bar fourteen on note F, and decrescendos down as the right hand approaches the note D. Other increases and decreases to volume follow through to the remainder of the A section, with the \textit{diminuendo} (a smooth decrease in volume) marker being the only other dynamic marker of note in the section.

\begin{figure}
  \centering
  \includegraphics[width=0.5\textwidth]{t-first-three-lines.jpg}
  \caption{The first three lines, in Tchaikovsky's June: Barcarolle}
  \label{fig:t-first-three-lines}
\end{figure}

\begin{figure}
  \centering
  \includegraphics[width=0.5\textwidth]{t-a-section-style-one.jpg}
  \includegraphics[width=0.5\textwidth]{t-a-section-style-two.jpg}
  \caption[Ornamentation Examples, in Tchaikovsky's June: Barcarolle]{Stylistic choices, written in to Tchaikovsky's June: Barcarolle}
  \label{fig:t-a-section-style-choices}
\end{figure}

\begin{figure}
  \centering
  \includegraphics[width=\textwidth]{t-b-section-bars-32-35.jpg}
  \caption{Bars 32-35 of Tchaikovsky's June: Barcarolle}
  \label{fig:t-b-section-bars-32-35}
\end{figure}

\begin{figure}
  \centering
  \includegraphics[width=0.7\textwidth]{t-b-section-bars-44-51.jpg}
  \caption{Bars 44-51 of Tchaikovsky's June: Barcarolle}
  \label{fig:t-b-section-bars-44-51}
\end{figure}


In the B section, the Venetian song continues, further immersing the listener into the song. The start of the section, as in Figure \ref{fig:t-b-section-bars-32-35}\autocite{Henle_2002}, begins with the tempo marking \textit{poco piú mosso} (roughly translating to ``a little bit faster''). It begins \textit{piano}, and as notated by the \textit{ma poco a poco crescendo}, the dynamic of the piece gradually increases slowly, until it reaches a peak at bar forty. The performer plays at \textit{forte}, indicating an increase in intensity within the Venetian gondolier's song, and a possible rockier nature to the gondola itself. This intensity is maintained through much of the section. Between bars 44-51, as in Figure \ref{fig:t-b-section-bars-44-51}\autocite{Henle_2002}, there are several other rhythmic and harmonic notations which the performer must keep in mind as well to maintain the necessary level of intensity and dramatization of the gondolier's song. Differently accented notes along with slurring certain lines contribute to the overall feeling of unsteadiness while the gondola floats its way down the Venice canal. Bars 44-46 are examples, as there are notes in both hands which begin as accented notes, and are followed immediately by slurred notes in short two-note phrases. Along with the staccato notes spread out in this figure, as well as the crescendo found in bar 47, the paddles of the gondola are heard in the background. The gondolier's singing is prominent above the sounds of the water in the canal and the rowing of the gondola itself, as the gondolier's singing becomes more dramatic and louder, through bar forty-seven's crescendo, and through to the arpeggiated chords in bars 50-51. The B section ends, with the arpeggiated chords of Figure \ref{fig:t-b-section-bars-44-51}\autocite{Henle_2002} becoming gradually slower with the \textit{poco ritenuto} tempo marking, culminating in a dramatic quarter-note length final note, and an extended quarter-note length rest (this extension is known as a fermata).

\begin{figure}
  \centering
  \includegraphics[width=0.5\textwidth]{t-transition-section.jpg}
  \caption[Transitioning between B section and A' section in Tchaikovsky's June: Barcarolle]{The transition section of Tchaikovsky's June: Barcarolle}
  \label{fig:t-transition-section}
\end{figure}

Slightly before the return of the A section, there is a brief measure and a half transition section. As in Figure \ref{fig:t-transition-section}\autocite{Henle_2002}, there is a descent is the harmonic line in the left hand, with bar 52 to be played in \textit{forte}, and the second beat of bar 53 played in \textit{mezzo forte}. This transition section reduces the intensity and high-energy that was found in the B section, moving it back to the gentle sensitivity of the A section. This is helped by the three note slur which starts this section, and the fermata over the two tied quarter notes which ends the section. Programmatically, the highest peak of the Venetian gondolier's song is over. The block chords of the B section give way for the broken chords style in the A section, with block chords in the transition section only signifying the transition to the A section. The gondola is able to continue on its journey, gentling floating down the Venice canals with the help of the gondolier.

\begin{figure}
  \centering
  \includegraphics[width=0.5\textwidth]{t-a-prime-beginning.jpg}
  \caption{A' Section in Tchaikovsky's June: Barcarolle}
  \label{fig:t-a-prime-beginning}
\end{figure}


Finally, the A' begins, featuring a return of many of the material that was treated in the first A section. The tempo of the piece goes back to \textit{Tempo I}, the A section's tempo and the original tempo of the piece. The dactylic nature of the A section also returns in A', and as in Figure \ref{fig:t-a-prime-beginning}\autocite{Henle_2002}, the Venetian gondolier's song has lessened in intensity and energy, with the listener only vaguely aware of the left hand's slurred motion which simulated the rocking of the gondola, and the gondolier slowly bringing the song to an end. Towards the end of the A' section, there is a return of the block chords which featured heavily in the B section, as in Figure \ref{fig:t-a-prime-ending}\autocite{Henle_2002}, starting with the marks which bracket the beginning of the ending in red. The Venetian gondolier's song, which is played in the right hand, decreases in volume to \textit{pianissimo} through a decrescendo, and turns to be a melody which is low in pitch and much more mysterious than it had been previously. The motion of the gondola also changes slightly, as the left hand's notes are no longer dactylic, beginning in measure 92. The left hand notes begin to be played in pairs, rather than groups of three, symbolizing the literal end of the gondola right which the listener is on. The block chords which end the piece in the right hand, from measures 92 to the end in Figure \ref{fig:t-a-prime-ending}\autocite{Henle_2002}, also signify that the gondolier's song is ending, and the listener is ready to disembark the gondola.

\begin{figure}
  \centering
  \includegraphics[width=0.8\textwidth]{t-a-prime-end-1.jpg}
  \includegraphics[width=0.8\textwidth]{t-a-prime-end-2.jpg}
  \caption{The ending of Tchaikovsky's June: Barcarolle}
  \label{fig:t-a-prime-ending}
\end{figure}

\section{A Gondolier's Song}

The June: Barcarolle is a character piece, meant to emulate the experience of being on a gondola ride through the canals of Venice, while a gondolier sings a folk-song. 

%Within these first five bars, we also take care to notice that the left hand's rhythm is reminiscent of a boat rocking
%
%Talk about how it's a character piece, so therefore is literally pulling in the motion of being on a gondola \& the Venetian gondolier singing
%
%Gloom atmostphere, also about death?
%
%Mix of happiness and sadness?
%
%Ending of the piece from bars 92-end we see the left hand be super two-note-y type where it's only one note then another, and not rocking motion like it was before. This could literally simulate the end of the boat ride, as the passengers disembark as the gondolier finishes singing.
%
%Insert somes interpretation about death \& acceptance.
