%!TEX root = ../username.tex
\chapter*{Preface}\label{pref}
\addcontentsline{toc}{chapter}{Preface}

Choosing the pieces which would make up my senior year recital was not the involved and difficult process that I had imagined it to be. In the summer before my senior year, I was tasked with finding pieces which would eventually make up my senior recital repertoire, and was encouraged to use a mixture of pieces I already knew, and pieces which were new. My ideal recital repertoire included pieces which I thought to be representative of the period of classical music in which the pieces originated. So, my journey to find pieces which represented the most notable features of the Baroque, Classical, Romantic, and Contemporary period began. This is the ``Big Four'' of sorts of the classical music periods (which entirely ignores the nuances between differences in sub-periods of classical music), and I found that there were a few key aspects of each period I wanted to emphasize in my repertoire. 

The Barque period of music (1600-1750) was influenced by certain ideas from the Enlightenment: order, logic, and reasoning. This results in a sound that is mathematically-sound, as it contains rhythms which are strict in adhering to the starting tempo of a piece, as well as a harmony and counterpoint technique. The harmony and counterpoint technique involves using multiple, simultaneous melodic lines which would move either together, or in separate directions, but always resolve to the tonic chord (the chord which is the key of the piece or movement). This culminates in a sound which lacks much emotion, and ultimately does not sound too complicated, as the music was written for the organ or harpsichord (the precursor to the modern day piano and pianoforte). 

The Classical period (1730-1820) is also strict in rhythm, clearly adhering to the established tempo, but also contains passages which are repeated twice. Typically, the first time a passage is played, it will be played \textit{forte} (loud) and the second time is \textit{piano} (soft) or vice versa, or the passage will be played with some other articulation difference between the first and the second time. These dynamic differences are possible due to the invention of the pianoforte, a novel instrument of the time. Dynamics were varied in volume much more than on the harpsichord. However, like the Baroque period, the Classical period also contains little emotion, resulting in the dynamic level differences of the Classical period to sound understated. Other features of the Classical period include elaborate melodies which are decorated by trills or other articulative tools, which sound over a broken chord accompaniment. Thus, the Classical era of music may contain music which has a slight dainty and delicate feeling.

The Romantic period of music (1820-1910) is music that sounds much more emotive than the first two periods of music. Most often, this means that performers and composers interpret Romantic era music to include \textit{rubato}, and to not interpret music to be in strict time. There is also a wide range of dynamic contrast, in which the music is able to tell a clear story, as there was the introduction of \textit{program music} during this era as well. To tell this story, or to paint a picture with the music, full chords (chords with contain more notes than triads) feature very heavily in music from this time, as do arpeggios (broken up chords) and the sustain pedal. Music thus feels ``bigger'' than music in previous eras might have, due both to the wider spacing of chords used, and the varied dynamic ranges present. 

Finally, the Contemporary period (other names include Modern/20th century, and generally said to take place between 1910-present day) of music rounds out the big four categories of classical music. This era of music often sounds quite dissonant, as other tuning methods and exploratory time and key signatures are used. Generally, Contemporary-era music sounds ``weird'' in comparison to other periods of music, with perhaps a jazz feeling to the music. Rhythms of Contemporary music are complicated, often with syncopation. 

The first piece of the program is Johann Sebastian Bach's \textit{Prelude and Fugue in C Minor}, from \textit{The Well-Tempered Clavier, Book I} BWV 847. The character of the prelude and fugue could not be more different, and this contrast caught my ear, as the \textit{Prelude in C Minor} begins fast with running sixteenth notes, and the \textit{Fugue in C Minor} has a lighter character beginning with only eighth notes. However, what drew me to this prelude and fugue pair the most was the clear demonstration of features which belonged to the Baroque period. I had a surface level understanding that the Baroque period of music drew many ideas from mathematical concepts and logic, but ultimately did no deep research in playing pieces of the Baroque era. So, I was hesitant to fully commit to these pieces at first, due to the expectations I had subconsciously set for myself. Before this year, I had always been drawn to playing either Classical- or Romantic-era music, as pieces from these times were ``easy'' to interpret, and I could figure out the tricks to make a good performance. But for music from the Baroque period, there are no ``tricks'' to create a performance. I have grown to appreciate the relative simplicity of the Baroque period, in which the music is absolute and tells no extra-musical story; instead, I focus on ensuring that each voice, as if in a SATB voicing, is clear to sound.

The second piece of the program is Wolfgang Amadeus Mozart's \textit{Piano Sonata in C Major}, No. 10, K. 330. This piece has enthralled me for years, as I have always found myself going back to part of the first movement as a warm-up, or to simply play through for fun. Learning the second and third movements of the piece, and thus becoming able to move fluidly through the three movements, was an experience I enjoyed. Much of the dynamic variance and articulative tools which are typical of music composed during the Classical period become clear at various points when playing this piece. I was excited to be able to take my affinity for music from the Classical era and put it into performance. The \textit{Piano Sonata in C Major}, No. 10 K. 330 in its entirety contains the dainty feeling typical of the Classical era, and to me will be the chance to play the piano as if it were a new creation of the past decade. 

The third piece of the program is Ludwig van Beethoven's \textit{Eleven Bagatelles}, Op. 119, No. 1. Beethoven as a composer straddles the line between the Classical period and the Romantic period, and so I believed that his work would serve as a respectable transition point between the rhythmically strict pieces of Bach and Mozart, and the free-flowing, expressive music of Chopin and Tchaikovsky. It is this uncertainty in placing Beethoven into one definitive category of Classical composer versus Romantic composer to which my ear was first drawn to when listening to the piece. First, this piece contains many of the features which make up the Classical era: strictness in tempo, passages are repeated twice, with some articulative differences in the second playback, and the use of trills to decorate the melody. But, the piece also contains aspects of the Romantic period, with an expressiveness that is rare to purely Classical-era music on display in the B section of the piece. Thus, I would be able to perform this piece as an ideal transition between two eras of music that have clear differences.

After the transition between the Classical and Romantic periods of music, I wanted to find pieces which were representative of the Romantic era. Many pieces of the time are grandiose in nature, and often reminiscent of other extra-musical ideas, with Edvard Grieg's \textit{In the Hall of the Mountain King} from Peer Gynt Suite No. 1, Op. 46, and Claude Debussy's ``Clair de lune,'' the third movement from Suite bergamasque, being two famous examples from the Romantic period. So, in searching for pieces which I felt were representative of the Romantic era (but that also were less well-known to the general public, I stumbled upon Frédéric Chopin's Prelude, Op. 28, No. 15, otherwise known as the ``Raindrop Prelude.'' This is still a famous piece out of the Romantic period, but to non-classical music fans would be unfamiliar. As many pieces of the Romantic era strove to do, this piece is designed to pain the picture of an extra-musical story, meant to be performed as if rain, and a rainstorm, were happening live. 

Pyotr Ilyich Tchaikovsky does similar things in his June: Barcarolle, one of the twelve pieces out of \textit{The Seasons}, Op. 37a. Like with the Raindrop Prelude, this Romantic-era piece tells the story of an event beyond itself. For this barcarolle, we are treated to an experience similar to those physically floating down the canals of Venice, through several prominent Romantic period composition tools which drew my ear to it during my first listen, which include the chordal harmonies and a depth to the piece's texture. The ability to perform slightly less popular pieces from the Romantic era provides me the ability to bring my own interpretation to the piece, regardless of how similar or different my interpretation may be from other performers. Thus, there is a certain freedom that is allotted to me as I perform a piece to which the general public, primarily composed of non-classical music fans, is unfamiliar. There is no comparison to how a piece is ``supposed'' to sound, as an audience may be prone to do if listening to pieces such as ``In the Hall of the Mountain King,'' or ``Clair de lune.''

The final piece of my senior recital program is Béla Bartók's \textit{Romanian Folk Dances}, Sz. 56, BB 68. Bartók is a composer out of the Contemporary period of classical music, and the final well-known period of classical music I wanted to be sure to include in my repertoire. Before undertaking this project, I had no exposure to composers nor pieces of the Contemporary period, having focused primarily on Baroque, Classical, and Romantic era music, along with the sub-periods of classical music. In my study of Bartók's music, I have found that is it easy to identify pieces which were composed during the Contemporary period, and pieces which were not. As mentioned, Contemporary-era pieces are solely ``weird'' in comparison to pieces from the remaining periods of classical music, with syncopated rhythms and a sound which overall sounds like it ``clashes'' to a listener's ear. However, as I have played the piece more, I have grown an appreciation for the features which Contemporary music offers. The dissonances found within each of the six dances somehow makes the purpose of each dance make more sense, through a type of compositional technique not allowed to composers before the twentieth-century. 