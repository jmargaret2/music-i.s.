\chapter[Beethoven and Eleven Bagatelles, Op. 119, No. 1]{Ludwig van Beethoven - \textit{Eleven Bagatelles,} Op. 119, No. 1 (1803)}\label{beethoven}

Ludwig van Beethoven (1770-1827) was a composer who rose to prominence towards the end of the Classical period. He was born in Bonn, in Northwest Germany, where both his father and grandfather were court musicians in Cologne. His works have been divided into three periods: his birth until approximately 1802, from 1803-1814, and finally from 1815 to his death in 1827 \autocite{Kerman_Tyson_Burnham_Johnson_Drabkin_2001}, reflecting stylistic changes and important life events. In the first period, Beethoven mastered the popular genres and musical concepts of his time and crafting his style. He studied with Joseph Haydn (another Classical era composer), and Johann Georg Albrechtsberger, learning the art of counterpoint. This led to compositions of wide-use, including virtuosic works and pieces for beginner piano students\autocite{Burkholder_Grout_Palisca_2014}, and private works for connoisseurs and public symphonies. Then in 1802, Beethoven experienced a gradual loss of hearing, which would render him fully deaf in a few years. With the support from patrons and sales to publishers, he was able to compose works of a new depth, reflected by the struggles he was facing in life. Finally, in the third period, his music became more introspective and difficult for performers to play and listeners to comprehend\autocite{Kerman_Tyson_Burnham_Johnson_Drabkin_2001}. By 1818, his hearing had worsened to near-deafness, and this prompted another change in style. This period featured a higher degree of contrast than before \autocite{Burkholder_Grout_Palisca_2014}, and exaggerated. There was contrast between style, figuration, meter, and tempo. To balance the contrasts, there was also an emphasis on continuity, blurring divisions between phrases or movements.

The bagatelle form is a short piece of music, sometimes defined to literally be \say{something short, a trifle}, and is composed to be light-hearted\autocite{Brown_2001}. The title \say{bagatelle} itself does not imply any specific musical form, and the term has been used as a generic title since Beethoven's three sets of bagatelles for piano (Opus 33, 119, and 126). His later two sets of bagatelles (Opus 119 and Opus 126) are anything but simple and a trifle. These two sets reflect the introspective second period of Beethoven, as his hearing declined. 

\section{Bagatelle, Op. 119, No. 1}
The first of the bagatelles in Beethoven's Opus 119, this bagatelle is in ternary form, a three-part musical form, typically notated as \textit{ABA`}. The final section, \textit{A`}, is a repeat of the first \textit{A} section. Each section of the form is self-contained, usually closing in its own key\autocite{Tucker_Cochrane_2011}. Thus, the A section will close in the tonic key, and the A` section will modulate to another, related key, and closes in that key.\footnote{This key will typically be the dominant or relative major key of the one in the first A section.} 